\documentclass{beamer}
\usetheme{Darmstadt}
\usecolortheme{default}

% balíčky

\usepackage[utf8]{inputenc}
\usepackage[czech]{babel}
\usepackage{graphicx}
\usepackage{mathtools}
\DeclarePairedDelimiter\ceil{\lceil}{\rceil}
\DeclarePairedDelimiter\floor{\lfloor}{\rfloor}



% informace o dokumentu

\title[krátký titulek prezentace]{Sundaramovo síto}
\subtitle[krátký název prezentace]{A srovnání se sítem Eratosthenovým}
\author{Autor: \textbf{Marco Souza de Joode}, G. Nad Štolou}

\date{\today}

% text dokumentu

\begin{document}

% titulní stránka
% (název prezentace, autor, datum...)

\begin{frame}
  \titlepage
\end{frame}

% osnova prezentace
% (většinou se pro délku vypouští)


% obsah prezentace
% (používají se klasické sekce)

\section{Teoretický úvod}
\begin{frame}

  \frametitle{Myšlenka}
  \framesubtitle{}
  % ...


Všechna přirozená čísla jsou buď sudá nebo lichá.
Všechna prvočísla, až na číslo 2, jsou lichá. \textbf{Sundaramovo síto} selektivně vybírá složená lichá čísla a odstraňuje je.	 

\end{frame}



\begin{frame}
  \frametitle{Součiny čísel podle parity}
 
Označíme-li si sudá čísla jako $S$ a lichá čísla jako $L$, pak platí
\begin{align*}
S\cdot S &= S\\
S \cdot L &= S\\
L \cdot S &= S\\
L \cdot L &= L
\end{align*}

protože
\begin{align*}
2n \cdot 2m &= 4nm = 2 \cdot 2nm\\
2n \cdot (2m+1) &= 4nm + 2n = 2 \cdot (2nm + n) \\
(2n +1) \cdot (2m+1) &= 4nm + 2n + 2m + 1 = 2 \cdot (n + m + 2nm) + 1
\end{align*}


\end{frame}




\begin{frame}
  \frametitle{Složená lichá čísla}

Všechna lichá čísla jsou buď prvočísla, nebo čísla složená. Každé liché složené číslo lze zapsat jako

\begin{align*}
&(2i + 1)(2j +1)\\
&= 4ij + 2i + 2j +1\\
&= 2(\underbrace{i + j + 2ij}_{U}) + 1
\end{align*}


\end{frame}



\begin{frame}
  \frametitle{Základní princip Sundaramova síta}

\begin{itemize}

\item Mějme $m = \floor{N/2}$
\item Pro všechna $i \in \{1, 2,..., m\}$ a pro všechna $j \in \{1, 2,..., m\}$ nalezněme množinu čísel $L$, které \textbf{nelze} zapsat jako $U = i+j+2ij$, když $U<m$.

\item Všechna lichá prvočísla lze zapsat jako $2U + 1$ pro každé $U$ v množině $L$.
\end{itemize}

\end{frame}




\begin{frame}
  \frametitle{Optimalizace Sundaramova síta}

\begin{itemize}

\item Mějme $m = \floor{N/2}$
\item Pro všechna $i \in \{1, 2,..., m\}$, aby $U = i+j+2ij < m$, musí platit omezení na $j$:

\begin{align*}
i+j+2ij &< m\\
j(2i+1) &< m-i\\
j &< \frac{m-i}{2i+1}
\end{align*}  
a zároveň nemusíme prověřovat $j \leq i$. Pak $j \in \{i, i+1,..., \frac{m-i}{2i+1}\}$

\item Všechna lichá prvočísla lze zapsat jako $2U + 1$ pro každé $U$ v množině $L$.
\end{itemize}

\end{frame}



\end{document}